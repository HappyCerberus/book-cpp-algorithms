\section{Heap data structure}

The standard offers a convenient wrapper for a max-heap data structure through \cpp{std::priority_queue}.
However, when using \cpp{std::priority_queue}, we lose access to the underlying data, which might be inconvenient.

\subsection{\texorpdfstring{\cpp{std::make_heap}, \cpp{std::push_heap}, \cpp{std::pop_heap}}{\texttt{std::make\_heap}, \texttt{std::push\_heap}, \texttt{std::pop\_heap}}}
\index{\cpp{std::make_heap}}
\index{\cpp{std::push_heap}}
\index{\cpp{std::pop_heap}}

A Heap data structure is a binary tree where each element satisfies the heap property: the value at the parent is greater or equal to the value of its children.

When discussing the heap algorithms, we will be referring to an array representation of the heap, where the children of the element at index $i$ are elements at indexes $2i+1$ and $2i+2$.

The valuable property of a heap is that it can be constructed in linear time and then provide logarithmic complexity for extracting the maximum value and inserting new values into the heap.

\cppversions{\texttt{make\_heap}}{\CC98}{\CC20}{N/A}{\CC20}
\cppversions{\texttt{push\_heap, pop\_heap}}{\CC98}{\CC20}{N/A}{\CC20}
\constraints{\texttt{random\_access\_range}}{}{\texttt{operator<}}{\texttt{strict\_weak\_ordering}}

The \cpp{std::make_heap} algorithm reorders elements in the given range such that the elements maintain the max-heap property, that is, the element at index $i$ compares greater or equal to the elements at indexes $2i+1$ and $2i+2$.

\begin{box-note}
\footnotesize Example of using \cpp{std::make_heap} to construct a max-heap and a min-heap (using a custom comparator).
\tcblower
\cppfile{code_examples/algorithms/make_heap_code.h}
\end{box-note}

The \cpp{std::push_heap} and \cpp{std::pop_heap} algorithms simulate push and pop operations for the max-heap data structure. However, because they operate on top of a range, they cannot manipulate the underlying data structure. Therefore, they use the last element of the range as the input/output.

The \cpp{std::push_heap} algorithm will insert the last element of the range into the heap, and \cpp{std::pop_heap} will extract the maximum element to the last position of the range. As previously mentioned, both operations have logarithmic complexity.

\begin{box-note}
\footnotesize Example of using \cpp{std::push_heap} and \cpp{std::pop_heap}.
\tcblower
\cppfile{code_examples/algorithms/push_heap_code.h}
\end{box-note}

\subsection{\texorpdfstring{\cpp{std::sort_heap}}{\texttt{std::sort\_heap}}}
\index{\cpp{std::sort_heap}}

The \cpp{std::sort_heap} will reorder the elements in a heap into a sorted order. Note that this is the same as repeatedly calling \cpp{std::pop_heap}; hence the algorithm has $O(n*logn)$ complexity.

\cppversions{\texttt{sort\_heap}}{\CC98}{\CC20}{N/A}{\CC20}
\constraints{\texttt{random\_access\_range}}{}{\texttt{operator<}}{\texttt{strict\_weak\_ordering}}

\begin{box-note}
\footnotesize Example of using \cpp{std::sort_heap}.
\tcblower
\cppfile{code_examples/algorithms/sort_heap_code.h}
\end{box-note}

\subsection{\texorpdfstring{\cpp{std::is_heap}, \cpp{std::is_heap_until}}{\texttt{std::is\_heap}, \texttt{std::is\_heap\_until}}}
\index{\cpp{std::is_heap}}
\index{\cpp{std::is_heap_until}}

The \cpp{std::is_heap} and \cpp{std::is_heap_until} algorithms check the heap invariant.

\cppversions{\texttt{is\_heap}}{\CC11}{\CC20}{N/A}{\CC20}
\cppversions{\texttt{is\_heap\_until}}{\CC11}{\CC20}{N/A}{\CC20}

\constraints{\texttt{random\_access\_range}}{}{\texttt{operator<}}{\texttt{strict\_weak\_ordering}}

The two algorithms follow the same logic as \cpp{std::is_sorted} and \cpp{std::is_sorted_until}, returning a boolean and an iterator to the first out-of-order element respectively.

\begin{box-note}
\footnotesize Example of using \cpp{std::is_heap} and \cpp{std::is_heap_until}.
\tcblower
\cppfile{code_examples/algorithms/is_heap_code.h}
\end{box-note}

\subsection{Comparison with \texorpdfstring{\cpp{std::priority_queue}}{\texttt{std::priority\_queue}}}
\index{\cpp{std::priority_queue}}

As mentioned at the beginning of this section, the heap algorithms provide effectively the same functionality as \cpp{std::priority_queue}. To demonstrate the differences, let's look at two versions of a topk algorithm: an algorithm that takes a range and returns the top k elements in sorted order as a \cpp{std::vector}.

\begin{box-note}
\footnotesize Example of implementing a topk algorithm using a \cpp{std::priority_queue}.
\tcblower
\cppfile{code_examples/extras/priority_queue_queue_code.h}
\end{box-note}

The example relies on \CC20 concepts to provide a generic interface and should therefore work with any range on input, returning a \cpp{std::vector} of the same element type.

When using the priority queue, we can utilize the simple \cpp{push()} and \cpp{pop()} interface provided (lines 12 and 14). However, extracting all data from the queue is only possible by repeatedly applying \cpp{pop()} until the queue is empty (line 20).

\begin{box-note}
\footnotesize Example of implementing a topk algorithm using heap algorithms.
\tcblower
\cppfile{code_examples/extras/priority_queue_heap_code.h}
\end{box-note}

When using the heap algorithms, we need to manually manage the underlying data structure (lines 9-10 and 13–14). However, we do not need to extract the data, and on top of that, we could omit the final \cpp{std::sort_heap} (line 20) if we do not need the top k elements in sorted order.